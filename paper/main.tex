\documentclass{article}
\usepackage{arxiv}

\usepackage[utf8]{inputenc}
\usepackage{lmodern}
\usepackage[english, russian]{babel}
\usepackage[T1]{fontenc}
\usepackage{url}
\usepackage{booktabs}
\usepackage{amsfonts}
\usepackage{nicefrac}
\usepackage{microtype}
\usepackage{lipsum}
\usepackage{graphicx}
\usepackage{natbib}
\usepackage{doi}



\title{Построение мультиагентной системы работы с ключевыми графиками научных статей}

\author{ Maksim M.~Mariasov \\
        Department of Mathematical Methods of Forecasting \\
        Moscow State University \\
        Lomonosov Moscow State University \\
        MSU Institute for Artificial Intelligence \\
	%% examples of more authors
	\And
	Roman V.~Ischenko \\
	Department of Mathematical Methods of Forecasting \\
        Moscow State University \\
        Lomonosov Moscow State University \\
        MSU Institute for Artificial Intelligence \\
	%% \AND
	%% Coauthor \\
	%% Affiliation \\
	%% Address \\
	%% \texttt{email} \\
	%% \And
	%% Coauthor \\
	%% Affiliation \\
	%% Address \\
	%% \texttt{email} \\
	%% \And
	%% Coauthor \\
	%% Affiliation \\
	%% Address \\
	%% \texttt{email} \\
}
\date{}

\renewcommand{\shorttitle}{Мультиагентная системы работы с ключевыми графиками научных статей}

%%% Add PDF metadata to help others organize their library
%%% Once the PDF is generated, you can check the metadata with
%%% $ pdfinfo template.pdf
\hypersetup{
pdftitle={A template for the arxiv style},
pdfsubject={q-bio.NC, q-bio.QM},
pdfauthor={David S.~Hippocampus, Elias D.~Striatum},
pdfkeywords={First keyword, Second keyword, More},
}

\begin{document}
\maketitle

\begin{abstract}
В работе исследуется способ построения мультиагентной системы автоматического выявления ключевых графиков научных статей, с целью повысить прозрачность и объяснимость процесса выбора графиков, что способствует эффективному анализу публикаций. Поскольку визуальная часть часто концентрирует в себе основные результаты статьи и сопровождает текстовые выводы, система, способная определять ключевые графики и анализировать причины их выбора, позволит исследователям быстрее и точнее выявлять наиболее значимые результаты публикации. Предлагается мультиагентная архитектура, объединяющая работу с текстовыми и визуальными признаками графиков с возможностью генерации объяснений, оценивающих вклад признаков в принятие решения. Проводится исследование отдельных агентов в зависимости от используемых подходов и архитектур и анализируется точность выбора на размеченном наборе статей и его согласованность с человеческими оценками, демонстрируя потенциал подхода.
\end{abstract}


\keywords{Explainable AI \and Key Figure Selection \and Multimodal Analysis \and Multi-agent Systems}

\section{Введение}
% Зачем это надо
Ключевые графики научных статей часто содержат основные результаты исследования и дополняют текстовые выводы, что делает их важным инструментом для быстрого понимания публикации \citep{Lee2016Viziometrics}. Выделение ключевых графиков требует изучение содержания статьи, что замедляет процесс понимания релевантности её для исследователя. Автоматизация процесса выбора ключевых графиков позволяет эффективнее анализировать литературу и быстрее выявлять значимые результаты. Более того, система с объяснимостью повышает доверие пользователей, делая процесс прозрачным.

% Как это решать
Существующие подходы для выявления ключевых графиков опираются на текстовые представления \citep{Beltagy2019SciBERT, Reimers2019SentenceBERT, scirus} или на визуальные представления графиков \citep{Radford2021CLIP, Li2022BLIP, Li2023BLIP2}. Мультимодальные модели объединяют информацию из текста и графиков, что позволяет улучшить качество выбора ключевых визуальных элементов. Для упорядочивания графиков по значимости применяются методы ранжирования, включая парный (pairwise) и списковый (listwise) подходы, рассмотренные в работе \citet{Liu2007LTR}. Для интерпретируемости решений используются методы Explainable AI, такие как Grad-CAM \citep{Selvaraju2017GradCAM}, Grad-CAM++ \citep{Chattopadhay2018GradCAMplusplus}, Score-CAM \citep{Wang2019ScoreCAM}, Text-CAM \citep{Zhang2025TextCAM} и LIME \citep{Ribeiro2016LIME}. Кроме того, применяются подходы рационализации \citep{Zhou2024IfCLIPCouldTalk, Chen2025ExplainableSaliency, Hendricks2018MultimodalExplanations}, которые позволяют объяснить, какие признаки текста или изображения оказали наибольшее влияние на решение модели. Использование мультиагентных систем обеспечивает модульность решения, независимость обработки разных модальностей и масштабируемость, облегчая интеграцию различных агентов для работы с текстом, изображениями и ранжированием.

% Какие проблемы решений
Существующие решения сохраняют ряд существенных ограничений. Во-первых, интеграция текстовой и визуальной модальностей остаётся сложной задачей вследствие различий в их структуры, что затрудняет построение согласованных представлений. Во-вторых, большинство методов автоматического определения ключевых графиков функционируют как «чёрные ящики», не предоставляя интерпретируемых объяснений, что ограничивает их практическую применимость и снижает доверие со стороны исследователей. В-третьих, модели, использующие крупные архитектуры для оценки межмодального сходства, характеризуются высокой вычислительной сложностью, особенно при масштабировании на большие наборы научных публикаций. Наконец, распространённые подходы анализируют графики изолированно, без учёта глобального контекста статьи и взаимосвязей между визуальными элементами, что приводит к снижению точности и устойчивости выбора ключевых графиков.

% Предлагаемое решение задачи
В данной работе предлагается мультиагентная система, включающая агента обработки текста (Text Agent) и агента обработки изображений (Image Agent), формирующих представления текста и графиков, а также агента ранжирования (Ranking Agent), который объединяет эти представления для окончательного выбора ключевых графиков. Агент объяснений (Explainable Agent) интерпретирует решения системы, демонстрируя вклад отдельных текстовых и визуальных признаков в выбор графиков. Такая архитектура позволяет использовать сильные стороны каждой модальности, при этом централизованно производя ранжирование и обеспечивая объяснимость процесса. Система способна адаптироваться к различным форматам научных статей и обеспечивает воспроизводимость выбора ключевых графиков.

% Какие результаты у того, что предложили
Предложенный подход демонстрирует высокую точность выбора ключевых графиков на размеченном наборе статей, показывая согласованность с оценками экспертов. Мультиагентная архитектура обеспечивает более стабильные и интерпретируемые результаты, позволяя проводить гибкие эксперименты с отдельными агентами. Агент объяснений предоставляет исследователям возможность визуально и текстово анализировать причины выбора графиков, повышая прозрачность и доверие к системе. В целом, предложенная система способствует автоматизации анализа научных публикаций и улучшает инструменты для выявления наиболее значимых результатов исследования.

\section{Обзор литературы}

\subsection{Представление текста}
Текстовые данные научных статей содержат ключевую информацию о результатах исследований. Аннотации, подписи к графикам и упоминания графиков в тексте могут содержать в себе признаки их значимости. Модели для извлечения текстовых эмбеддингов, такие как \textbf{SciBERT} \citep{Beltagy2019SciBERT}, \textbf{Sentence-BERT} \citep{Reimers2019SentenceBERT} и \textbf{SciRus} \citep{scirus}, позволяют формировать векторные представления текста, которые применяются для оценки значимости графиков и их связи с основными результатами статьи.

\subsection{Представление изображений}
Графики и диаграммы отражают результаты статьи, поэтому методы, опирающиеся только на текстовые данные не способны быть полными. Для извлечения признаков изображений применяются модели \textbf{CLIP} \citep{Radford2021CLIP}, \textbf{BLIP} \citep{Li2022BLIP} и \textbf{BLIP-2} \citep{Li2023BLIP2}, которые создают компактные представления графиков и позволяют сравнивать их с текстовыми описаниями. 

\subsection{Мультимодальные представления}
Объединение текстовой и визуальной информации повышает точность выявления ключевых графиков. Мультимодальные модели, включая BLIP \citep{Li2022BLIP}, BLIP-2 \citep{Li2023BLIP2}, \textbf{LamRA} \citep{Liu2025LamRA} и \textbf{PaLI-3} \citep{Chen2023PaLI3}, объединяют представления текста и изображений в единое пространство, учитывая взаимосвязь между подписью и визуальными особенностями графика, что позволяет эффективно анализировать связи между двумя модальностями.

\subsection{Ранжирование}
Для упорядочивания графиков по значимости применяются методы \textbf{Learning-to-Rank}: парный (pairwise) и списковый (listwise) подходы \citep{Liu2007LTR}. Также рассматриваются методы на основе обучения с подкреплением (RL) \citep{Wang2024MultimodalRL} и нейросетевые ранжирующие модели (MLP, Poly-encoders \citep{Humeau2019PolyEncoders}), которые учитывают сложные взаимосвязи между текстовыми и визуальными признаками.

\subsection{Explainable AI и рационализации}
Интерпретируемость решений моделей критична для доверия пользователей. Для визуальных моделей применяются методы \textbf{Grad-CAM} \citep{Selvaraju2017GradCAM}, \textbf{Grad-CAM++} \citep{Chattopadhay2018GradCAMplusplus}, \textbf{Score-CAM} \citep{Wang2019ScoreCAM} и \textbf{Text-CAM} \citep{Zhang2025TextCAM}. Для текста и мультимодальных решений используются \textbf{LIME} \citep{Ribeiro2016LIME} и подходы рационализации \citep{Zhou2024IfCLIPCouldTalk, Chen2025ExplainableSaliency, Hendricks2018MultimodalExplanations}, позволяющие показать, какие признаки текста или изображения оказали наибольшее влияние на выбор ключевых графиков.

\section{Постановка задачи}

%\section{Предложенный  метод}

\section{Эксперименты}

\section{Заключение}

\bibliographystyle{unsrtnat}
\bibliography{references}

\end{document}